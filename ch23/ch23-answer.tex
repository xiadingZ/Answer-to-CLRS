
%%% Local Variables:
%%% mode: latex
%%% TeX-master: t
%%% End:
\documentclass{article}
\usepackage{amsmath}
\usepackage{algorithm}
\usepackage{algorithmic}


\author{Xia Ding}
\title{\textbf{Introduction To Algorithm}\\Third Edition\\\textbf{Answer}}

\begin{document}
\maketitle
\section*{23.1}
\subsection*{23.1--1}
Theorem 23.1 in book shows this. Let A be the empty set and S be any set
containing $u$ but not $v$.

\subsection*{23.1--2}
Suppose we have a graph G with many edges, there is a node with which only
one edge $e$ is incident. $e$ must be a safe edge for A, but it may be not a
light edge for some cut if it has a very high weight.

\subsection*{23.1--3}
Let (S, V$-$S) be a cut such that $u\in{S}$ and $v\in{V-S}$, because $(u, v)$
is contained in some minimum spanning tree, so it is a light edge crossing
this cut.

\subsection*{23.1--4}
A triangle whose edge weights are all equal is a graph in which every edge
is a light edge crossing some cut. But the triangle is cyclic, so it's not
a minimum spanning tree.

\subsection*{23.1--5}
Because $e$ is in some cycle, every nodes in that cycle should be in some
minimum spanning tree T. Let that cycle be a cycle in which no other cycle exist. So some edge in that cycle can't be in T, marked it as $e_1$. If $e_1 = e$,
we are done, if not, suppose $e$ is in some minimum spanning tree $T_1$, then
we can construct a $T_2$ such that $w(T_2) \le w(T_1)$ by removing $e$ and add
$e_1$ because $w(e_1) \le w(e)$. So there is a minimum spanning tree of G that
doesn't include $e$.

\subsection*{23.1--6}
Suppose that for every cut of G, there is a unique light edge crossing the cut
. Let us consider two minimum spanning trees, $T$ and $T'$, of G. We will show
that every edge of $T$ is also in $T'$, which means that $T$ and $T'$ are the
same tree and hence there is a unique minimum spanning tree.\\
Consider any edge $(u, v)\in{T}$. If we remove $(u, v)$ from $T$, then $T$ becomes disconnected, resulting in a cut $(S, V-S)$. The edge $(u, v)$ is a light
edge crossing the cut $(S, V-S)$(by Exercise 23.1-3). Now consider the edge
$(x, y)\in{T'}$ that cross $(S, V-S)$ is unique, the edges $(u, v)$ are the
same edge. Thus, $(u, v)\in{T'}$. Since we chose $(u, v)$ arbitrarily, every
edge in $T$ is also in $T'$.

A counter example is such G that $V = {x, y, z}$, $E = {(x, y, 1), (x, z, 1)}$(\textit{$(a, b, w)$ is a edge $(a, b)$ weighted $w$}). Consider the cut
$({x}, {y, z})$. Both of the edges $(x, y)$ and $(x, z)$ are light edges
crossing the cut, and they are both light edges.

\subsection*{23.1--7}
According to the definition, if all edge weights are positive, then we can't
have a cycle in minimum spanning graph, because if exist, we can always remove
an edge without removing nodes from the minimum spanning graph to decrease the
total weight. So any subset of edges that connects all vertices and has
minimum total weight must be a tree.

If we allow some weights to be non-positive, then we can construct a cycle
without increasing the total weight if that cycle contains edges weighted
non-positive, so that some subset of edges that connects all vertices and has
minimum total weight can't be a tree.

\subsection*{23.1--8}


\subsection*{23.1--9}
If $T'$ isn't a minimum spanning tree of $G'$, then there exists a path
$ve_1e_2\ldots{}e_nu$ to replace original $(v, u)$ in $T'$ so that total weightof $T'$ will decrease. Because $T'$ is a subset of $T$, so $(v, u)$ is in $T$
too, then we can use $ve_1e_2\ldots{}e_nu$ to replace $(v, u)$ in $T$ so that
total weight of $T$ will decrease, contradicting that $T$ is a minimum
spanning tree.

\subsection*{23.1--10}
Suppose we make initial cut as $(\empty, V-\empty)$, then after we construct
a MST with m edges, then they are m minimum weight edges of G and $(x, y)$ is
in them. After decrease $(x, y)$'s weight, those edges are still m minimum weight edges in G, so $T$ is still a MST of G.

\subsection*{23.1--11}


\section*{23.2}
\subsection*{23.2--1}

\subsection*{23.2--2}
Algorithm ..
\begin{algorithm}
  \caption{MATRIX-PRIM($G, w, r$)}
  \begin{algorithmic}[1]
    \FORALL {$u \in{G.V}$}
    \STATE $u.key = \infty$
    \STATE $u.\pi = $ NIL
    \ENDFOR
    \STATE $r.key = 0$
    \STATE $Q = G.V$
    \WHILE {$Q \ne \emptyset$}
    \STATE $u =$ EXTRACT-MIN(Q)
    \FOR {$v = 1$ to $|V|$}
    \IF {$G.A[u, v] \ne 0$ \AND $j \in{Q}$ \AND $w(u, v) < v.key$}
    \STATE $v.\pi = u$
    \STATE $v.key = w(u, v)$
    \ENDIF
    \ENDFOR
    \ENDWHILE
  \end{algorithmic}
\end{algorithm}

\subsection*{23.2--3}
TODO

\subsection*{23.2--4}
TODO

\subsection*{23.2--5}
TODO

\subsection*{23.2--6}
TODO

\subsection*{23.2--7}
TODO

\subsection*{23.2--8}
TODO

\section*{Problems}
\subsection*{23--1}
TODO

\subsection*{23--2}
TODO

\subsection*{23--3}
TODO

\subsection*{23--4}
TODO

\end{document}