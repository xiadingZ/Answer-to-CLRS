
%%% Local Variables:
%%% mode: latex
%%% TeX-master: t
%%% End:
\documentclass{article}
\usepackage{amsmath}
\usepackage{algorithm}
\usepackage{algorithmic}


\author{Xia Ding}
\title{\textbf{Introduction To Algorithm}\\Third Edition\\\textbf{Answer}}

\begin{document}
\maketitle
\section*{22.1}
\subsection*{22.1--1}
Both of them take $O(|V|)$ time.

\subsection*{22.1--2}
%%TODO

\subsection*{22.1--3}
Algorithm 1 and 2. Transpose of matrix takes time $O(V^2)$. Transpose of list
takes time $O(V+E)$.
\begin{algorithm}
  \caption{TRANSPOSE-ADJ-MATRIX(G)}
  \begin{algorithmic}[1]
    \STATE $G^T = G$
    \FOR {$i = 1$ to $|V|$}
    \FOR {$j = 1$ to $i$}
    \STATE exchange $G^T.A[i, j]$ with $G^T.A[j, i]$
    \ENDFOR
    \ENDFOR
    \RETURN $G^T$
  \end{algorithmic}
\end{algorithm}

\begin{algorithm}
  \caption{TRANSPOSE-ADJ-LIST(G)}
  \begin{algorithmic}[1]
    \FOR {$i = 1$ to $|V|$}
    \FOR {$j$ in $G.Adj[i]$}
    \STATE INSERT$(G^T.Adj[j], i)$
    \ENDFOR
    \ENDFOR
    \RETURN $G^T$
  \end{algorithmic}
\end{algorithm}

\subsection*{22.1--4}
Algorithm 3.\\
We use an auxiliary matrix A size of $|V|\times|V|$ to identify whether there is a edge between \textit{v, w}.
\begin{algorithm}
  \caption{ELIMINATE-MULT(G)}
  \begin{algorithmic}[1]
    \STATE A = CREATE-MATRIX$(|V|, |V|)$
\COMMENT{create a $|V|\times|V|$ empty matrix}
    \FOR {$i = 1$ to $|V|$}
    \FOR {$j in G.Adj[i]$}
    \IF {$A[i, j] == 0$ \AND $A[j, i] == 0$}
    \STATE Insert$(G'.Adj[i], j)$
    \STATE Insert$(G'.Adj[j], i)$
    \ENDIF
    \STATE $A[i, j] = 1$
    \STATE $A[j, i] = 1$
    \ENDFOR
    \ENDFOR
    \RETURN $G'$
  \end{algorithmic}
\end{algorithm}

\subsection*{22.1--5}
Algorithms 4 and 5.
\begin{enumerate}
\item For adjacency-matrix, square of a graph is square of it's adjacency-matrix. Use Strassen's algorithm to multiple matrix, running time is $O(|V|^{2.8})$. Then scan the result to set the non-zero number to 1. Total time is
$O(|V|^{2.8})$.
\item For adjacency-list. Use a auxiliary $|V|\times|V|$ matrix, running time
is $O(V+E^2)$
\end{enumerate}

\begin{algorithm}
  \caption{SQUARE-GRAPH-MATRIX(G)}
  \begin{algorithmic}[1]
    \STATE $G^2.A$ = MATRIX-MULTIPLE(G.A, G.A) \COMMENT{Strassen's algorithm}
    \FOR {$i = 1$ to $|V|$}
    \FOR {$j = 1$ to $|V|$}
    \IF {$G^2.A[i, j] \ne 0$}
    \STATE $G^2.A[i, j] = 1$
    \ENDIF
    \ENDFOR
    \ENDFOR
  \end{algorithmic}
\end{algorithm}

\begin{algorithm}
  \caption{SQUARE-GRAPH-LIST(G)}
  \begin{algorithmic}[1]
    \STATE A = CREATE-MATRIX$(|V|, |V|)$
    \FOR {$i = 1$ to $|V|$}
    \FORALL {$j$ in $G.Adj[i]$}
    \FORALL {$k$ in $G.Adj[j]$}
    \IF {$A[i, j] == 0$}
    \STATE INSERT($G^2.Adj[i], j$)
    \STATE $A[i, j] = 1$
    \ENDIF
    \IF {$A[i, k] == 0$}
    \STATE INSERT$(G^2.Adj[i], k)$
    \STATE $A[i, k] = 1$
    \ENDIF
    \ENDFOR
    \ENDFOR
    \ENDFOR
  \end{algorithmic}
\end{algorithm}

\subsection*{22.1--6}
Algorithm 6.\\
\begin{algorithm}
  \caption{UNIVERSAL-SINK(G)}
  \begin{algorithmic}[1]
    \STATE TODO
  \end{algorithmic}
\end{algorithm}

\subsection*{22.1--7}
\begin{displaymath}
  BB^T(i, j) = \sum_{e\in{E}}b_{ie}b_{ej}^T = \sum_{e\in{E}}b_{ie}b_{ej}
\end{displaymath}

\begin{itemize}
\item If $i = j$, then $b_{ie}b_{je} = 1$(it's $1\cdot1$ or $(-1)\cdot(-1)$)
whenever e enters or leaves vertex $i$, and 0 otherwise.
\item If $i \ne j$, then $b_{ie}b_{je} = -1$ when $e = (i, j)$ or $e = (j, i)$,
and 0 otherwise.
\end{itemize}
Thus,\\
\begin{displaymath}
  BB^T  = \left\{ \begin{array}{lr}
\textrm{degree of} i = \textrm{in-degree + out-degree} & \textrm{if}~i = j\\
-(\textrm{number of edges connecting}) i~\textrm{and}~j& \textrm{if}~ i \ne j
\end{array} \right.
\end{displaymath}

\subsection*{22.1--8}
\begin{enumerate}
\item It takes $O(1)$ time.
\item Disadvantage is wasting space or doesn't save much time when
searching a node's out-vertices.
\item Using binary search tree. Can save time when determining whether an
edge is in the graph, and doesn't waste space, compared to linked list
representation, but use more time compared to hash table.
\end{enumerate}

\section*{22.2}
\subsection*{22.2--1}
It's easy to do it by yourself.

\subsection*{22.2--2}
It's easy to do it by yourself.

\subsection*{22.2--3}
If the bit is 0, then that node hasn't been visited, if is 1, it has been
visited. So we can substitute a bit for color attribute.

\subsection*{22.2--4}
Running time is $O(|V|^2)$, because we must scan every elements in matrix.

\subsection*{22.2--5}
The first time we visit $u$ from $v$, we set $u.d = v.d + 1$, independent of the order in which $u$ appears in $v$'s adjacency list. After that, $u.d$
won't change.

\end{document}