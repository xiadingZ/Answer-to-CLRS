\documentclass{article}
\usepackage{amsmath}
\usepackage{tikz}
\usepackage{algorithm}
\usepackage{algorithmic}


\author{Xia Ding}
\title{\textbf{Introduction To Algorithm}\\Third Edition\\\textbf{Answer}}

\begin{document}
\maketitle
\section*{5.1}
\subsection*{5.1--1}
\begin{description}
\item[Partial order:] A relation which is reflexive, anti-symmetric and
  transitive.
\item[Total order:] A total order that is a partial order which has
  \textit{totality} property\\($\forall a~b \in S$: aRb or bRa).
\end{description}
First, we prove the partial order.
\begin{itemize}
\item \textit{Reflexive}: Because everybody is as good or better as themselves.
\item \textit{Transitive}: If A is better than B and B is better than C, then A
  is better than C.
\item \textit{Anti-symmetric}: If A is better than B, than B is not better than A.
\end{itemize}
 Now, the set S is partial order, because any two candidates are comparable, so
 the set S is total order.

\subsection*{5.1--2}
Let $n = b - a$. The algorithm is as follows:
\begin{enumerate}
\item find the smallest integer $c$ such that $2^c \ge n$, that is $c =
  \lceil{\lg n}  \rceil$.
\item call RANDOM (0, 1) $c$ times to get a $c$-digit binary number $r$.
\item if $r > n$ we go back to the step 2.
\item return $a + r$.
\end{enumerate}
This produces a uniformly random number in that range. However, there is a
possibility to have to repeat step 2. There is $p=\frac{n}{2c}$ chance of not
having to repeat step two. The geometric distribution suggests that on average
it takes $\frac{1}{p}$ trials before we get such a number, that is
$\frac{2c}{n}$ trials. Since we perform $c$ calls to RANDOM (0, 1) on each trial,
the expected running time is
\[O(\frac{2^c}{n}c) = O(\frac{\lg (b-2)2^{\lg (b-a)}}{b-a}) = O(\lg (b-a))\]

\subsection*{5.1--3}
\begin{enumerate}
\item Generate two random numbers $x$ and $y$
\item If they are not same, return $x$
\item Otherwise, repeat from step 1
\end{enumerate}
Because $P[0, 1] = P[1, 0] = 2p(1-p)$, so number of trails is
$\frac{1}{2p(1-p)}$. So running time is $O(\frac{1}{p(1-p)})$.


\section*{5.2}
\subsection*{5.2--1}
When the best candidate is the first one fired, then you hire exactly one time,
$P = \frac{(n-1)!}{n!} = \frac{1}{n}$.  When the order of candidates is
increasing, you will hire $n$ times, $P = \frac{1}{n!}$.

\subsection*{5.2--2}
You hire twice when your first fired candidate is rank $i(i < n)$ and all candidates
with rank $k$ such that $i < k < n$ come after candidate with rank $n$.  Let
$X_i$ be the case that first candidate is rank $i$ and you hire exactly twice.
Because you first hire candidate ranked $i$ has probability $\frac{1}{n}$, and
the best candidate comes first in rest $n - i$ better candidate has probability
$\frac{1}{n - i}$. So, 
\[P(X_i) = \frac{1}{n}\frac{1}{n-1}\]
So the probability of hiring exactly twice is:
\[P(X) = \sum_{i = 1}^{n-1}P(X_i) = \frac{1}{n}\sum_{i = 1}^{n-1}\frac{1}{i}
  = \frac{1}{n}(\lg(n - 1) + O(1))\]

\subsection*{5.2--3}
First, we analysis one dice. Let $X_i $ be the indicator such that $X_i = I{this
dice shows number i}$. Let $Y_j$ be the random variable denoting the
number got from the $j-th$ dice. Then $Y_j = \sum_{i = 1}^6X_i$, because $E[X_A] =
\Pr\{A\}$,
\begin{eqnarray*}
   E[Y_j] &=& E[\sum_{i = 1}^6X_i] \\
        &=& \sum_{i = 1}^6E[X_i]\\
        &=& \sum_{i = 1}^6i\Pr\{i\}\\
        &=& \sum_{i = 1}^6i\frac{1}{6}\\
        &=& 3.5
\end{eqnarray*}
 So the expect value of sum of $n$ dice is:
\[E[Y] = E[\sum_{j = 1}^n Y_j] = \sum_{j = 1}^n3.5 = 3.5n\]

\subsection*{5.2--4}
Define a random variable $X$ that equals the number of customers that get back their
own hat, so that we want to compute $E[X]$.
For $i = 1, 2, 3\ldots{}n$, define the indicator random variable\\
$X_i = I\{customer i gets back his own hat\}$.\\
Then $X = X_1 + X_2 + \cdots{} + X_n$.\\
Since the ordering of hats is random, each customer has a probability of $1/n$
of getting back his hat. In other word, $\Pr\{X_i = 1\} = 1/n$, which, implies
that $E[X_i] = 1/n$. Thus:
\[E[X] = E[\sum_{i = 1}^n X_i] = \sum_{i = 1}^n\frac{1}{n} = 1\]

\subsection*{5.2--5}
Let $X_{ij}$ be an indicator random variable for the event where the pair $A[i],
A[j]$ for $i < j$ is inverted, i.e., $A[i] > A[j]$. More precisely, we define
$X_{ij} = I{A[i] > A[j]}$ for $1 \le i < j \le n$. We have $\Pr{X_{ij} = 1} =
1/2$, because given two distinct random numbers, the probability that the first
is bigger than the second is 1/2, By Lemma 5.1, $E[X_{ij} = 1/2]$.\\
Let $X$ be the random variable denoting the total number of inverted pairs in
the array, so that
\[X = \sum_{i = 1}^{n-1}\sum_{j = i + 1}^n\!\!X_{ij}\]
We want the expected number of inverted pairs, so we take the expectation of
both sides of the above equation to obtain
\begin{eqnarray*}
E[X] &=& E[\sum_{i = 1}^{n - 1}\sum_{j = i + 1}^n\!\!X_{ij}]\\
  &=& \sum_{i = 1}^{n - 1}\sum_{j = i + 1}^n \frac{1}{2}\\
  &=& \binom{n}{2}\frac{1}{2}\\
  &=& \frac{n(n-1)}{4}
\end{eqnarray*}


\section*{5.3}
\subsection*{5.3--1}
Before enter the loop, you can pick a random number in A and swap it with A[1].
Then start the loop with $i = 2$. And modify the Proof's
\textbf{Initialization:} to ``Consider the situation just before the first loop
iteration, so that $i = 2~~\cdots{}$.''

\subsection*{5.3--2}
Although it will not produce the identity permutation there are other
permutations that it fails to produce. For example, consider
its operation when $n = 3$, when it should be able to produce the $n!-1 = 5$  
non-identity permutations. The for loop iterates for $i = 1$ and $i = 2$. When
$i = 1$, the call to RANDOM returns one of two possible values (either 2 or 3),
and when $i = 2$, the call to RANDOM returns just one value (3). Thus, this
procedure  can produce only 2 possible permutations, rather than the 5 that
are required.

\subsection*{5.3--3}
No. Because it produce $n^n$ different permutations, not $ n! $.

\subsection*{5.3--4}
Because once \textit{offset} is determined, the entire permutation is
determined. And B is the result of shifting A, each value of \textit{offset}
occurs with probability $1/n$, so $A[i]$ has a $1/n$ probability of winding up
in any particular position in B. However, there are only n different
permutations, not $ n! $, so the result is not uniformly random.

\subsection*{5.3--5}
\begin{eqnarray*}
  P &=& \prod_{i = 0}^{n-1}\frac{n^3-i}{n^3} \\
    &\ge& {(1 - \frac{1}{n^2})}^n \\
    &\ge& 1 - \frac{1}{n}  \qquad (Bernoulli's \quad inequality)
\end{eqnarray*}

\subsection*{5.3--6}
Generate new properties and retry.

\subsection*{5.3--7}
We use a loop invariant:
\begin{quote}
  RANDOM-SAMPLE$(i, n-m+i)$ produce a random $i$-subset $S$ of
  \{1,2,3,$\ldots{}n-m+i$\} for $i = 0,1,2,\ldots{}m$
\end{quote}
\begin{description}
\item[Initialization:] When $i = 0$, RANDOM$(0, n-m)$ return $\emptyset$ with
  probability 1, which is $\binom{n-1}{0}$.
\item[Maintenance:] We assume RANDOM$(i-1, n-m-1+i)$ produce a random
  \mbox{$(i-1)$-subset} of \{1,2,3,$\ldots{}n-m-1+i$\}, with each possible result have
  probability $1/\binom{n-m-1+i}{i-1}$. Then after call RANDOM$(i, n-m+i)$, the
  probability of $i$-subset containing $n-m+i$ is 
\[\frac{i}{n-m+i}\frac{1}{\binom{n-m-1+i}{i-1}} = \frac{1}{\binom{n-m+i}{i}}\]
If it doesn't contain $n-m+i$, it may contain one of $n-m$ numbers, probability of
each is:
\[\frac{n-m}{n-m+i}\frac{1}{\binom{n-m-1+i}{i}} = \frac{1}{\binom{n-m+i}{i}}\]
So, RANDOM$(i, n-m+i)$ produce a random random $i$-subset $S$ of
\{1,2,3,$\ldots{}n-m+i$\}
\item[Termination:] At termination, $i = m$, and we conclude that RANDOM$(m, n)$
  produce a random $m$-subset of \{1,2,3,$\ldots{}n$\}.
\end{description}

\section*{5.4}
\subsection*{5.4--1}

\section*{Problems}
\subsection*{5.1 \textit{Probabilistic counting}}
\begin{description}
\item[a.] Suppose at the start of the $m-th$ INCREMENT OPERATION, the value of counter is
$i$, if $i$ increase, the number represented by $i$ will increase by
$n_{i+i}-n_i$. We use $X_m$ be the value of increment of number represented by
$i$.
\begin{eqnarray*}
E[X_m] &=& 0\Pr\{i~don't~increase\} + (n_{i+1}-n_i)\Pr\{i~increase\}\\
       &=& 0\cdot(1-\frac{1}{n_{i+1}-n_i}) + (n_{i+1}-n_i)\cdot\frac{1}{n_{i+1}-n_i}\\
       &=& 1
\end{eqnarray*}
The expected value of increment of number represented by $i$ is 1 during each
INCREMENT operation, so after $n$ INCREMENT OPERATION, expected value
represented by counter is exactly $n$.

\item[b.] 
  \begin{eqnarray*}
    Var[X_m] &=& E[X_m^2] - E^2[X_m]\\
             &=& 0^2\cdot\frac{99}{100}+100^2\cdot\frac{1}{100} - 1^2\\
             &=& 99
  \end{eqnarray*}
So variance of single INCREMENT operation is 99, after $n$ INCREMENT operation, 
\[Var[X] = Var[\sum_{i = 1}^nX_i] = \sum_{i = 1}^nVar[X_i] = 99n\]
\end{description}


\subsection*{5.2 \textit{Searching an unsorted array}}
\begin{description}
\item[a.] Algorithm 1.
\begin{algorithm}
  \caption{RANDOM-SEARCH$(A, x)$}
  \begin{algorithmic}[1]
    \STATE $S = \emptyset$ \COMMENT{S is a set}
    \WHILE {$|S| \ne n$}
    \STATE $i = \textrm{RANDOM}(1, n)$
    \IF {$A[i] \ne x$}
    \STATE $S = S \cap i$
    \ELSE 
    \RETURN $i$
    \ENDIF 
    \ENDWHILE
    \RETURN None
  \end{algorithmic}
\end{algorithm}

\item[b.] Because probability of getting $i$ is $1/n$, so the expected value of
trails is $n$.

\item[c.] Because probability of getting $i$ is $k/n$, so the expected value of
trails is $n/k$.

\item[d.] According to balls and bins model in 5.4.2, expected value is
  $n(\ln n+O(1))$.

\item[e.] The worst-case running time is $n$. The average-case is $(n+1)/2$.
\newpage
\item[f.] The worst-case running time is $n-k+1$. Let $X_i$ be an indicator
  random variable representing $A[i] = x$. $\Pr\{X_i\} = \frac{1}{k+1}$. Let Y be
  the indicator random variable representing after $n-k$ search, we find $A[i]
  = x$. $\Pr{Y} = 1$.
\[E[X] = E[X_1+X_2+X_3+\cdots+X_{n-k}+Y] = 1+\frac{n-k}{k+1} = \frac{n+1}{k+1}\]

\item[g.] Both of them are $n$.

\item[h.] The same as solution of part (\textbf{f}), replacing ``average-case''
  with ``expected''.

\item[i.] DETERMINISTIC-SEARCH. Because SCRAMBLE-SEARCH will first permute the
  array randomly, which will cost $O(n)$ time.
\end{description}

\end{document}